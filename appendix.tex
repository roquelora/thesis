%
%  Example Appendix pages.
%  Modified to use new usu-thesis-mk2 appendix facilities.
%
%  Time-stamp: "[appendix.tex] last modified by Scott Budge (scott) on 2011-08-08 (Monday, 8 August 2011) at 15:46:06 on goga"
%
%  Info: $Id$   USU
%  Revision: $Rev$
% $LastChangedDate$
% $LastChangedBy$
%
%
% For a single appendix, use \makeappendix, and place the 
% body of the appendix after it

%\makeappendix

% < single appendix body here >

% For multiple appendices, use \makeappendices, and create each appendix
% using \appendix{}
% For sub-appendices use \appendixsection{} and \appendixsubsection{}

\makeappendices
\appendix{List of Edge Vectors}
\label{chap:appendix}


\appendixsection{Definition of an Edge Vector}
\label{sec:edge-def}

Before we list the table of edge vectors, we need to describe what an
edge vector is.  In this section we will describe in detail the theory
that results in the edge vectors.  The first set of edge vectors is
given in Table~\ref{table1}.

\begin{table}[!t]
% increase table row spacing, adjust to taste
  \renewcommand{\arraystretch}{1.3}

  \caption{List of edge vectors for a codebook with b=8 and d=3, for a
    $4 \times 4$ vector size.}
  \label{table1}

  \centering
  \begin{tabular}{|c|c|} \hline
    Level & Edge Vectors \\ \hline

    & (5) \\ 
    L1 & (6) \\
    & (7) \\ \hline

    & (3,1)\\
    & (3,2)\\
    & (3,5)\\
    & (4,0)\\
    L2 & (4,2)\\
    & (4,3)\\
    & (4,4)\\
    & (4,5)\\
    & (4,6) \\ \hline

    & (3,4,1)\\
    & (3,4,2)\\
    & (3,7,0)\\
    & (3,7,2)\\
    & (3,7,4)\\
    L3 & (4,1,0)\\
    & (4,1,1)\\
    & (4,1,2)\\
    & (4,1,3)\\
    & (4,1,4)\\
    & (4,1,5)\\
    & (4,1,6) \\ \hline

  \end{tabular}

\end{table}


\appendixsection{Next Codebook Size Description}
\label{sec:next-size}

In this section we do the next size codebook.  This is different from
the previous case in that the codebook size is different.  The next
set of edge vectors is given in Table \ref{table2}.

\appendixsection{Final Set of  Codebook Size Descriptions}
\label{sec:final-size}

The following three tables contain the data for codebook sizes that
are different than the previous sizes.  We note that the differences
in the tables are due to the differences in the sizes of the codebook
edge vectors.  Note the values given in Table \ref{table3} --
Table \ref{table5}.


\begin{table}[!t]
  \renewcommand{\arraystretch}{1.3}
  \centering

  \caption{List of edge vectors for a codebook with b=4 and d=3, for a
    $4 \times 4$ vector size.}
  \label{table2}

  \begin{tabular}{|c|c|} \hline
    Level & Edge Vectors \\ \hline

    & (1)\\
    L1 & (2)\\
    & (3) \\ \hline

    L2 & (0,3) \\ \hline

    & (0,2,0)\\
    L3 & (0,2,2)\\
    & (0,2,3) \\ \hline

  \end{tabular}

\end{table}

\begin{table}[!t]
  \renewcommand{\arraystretch}{1.3}
  \centering

  \caption{List of edge vectors for a codebook with b=16 and d=3, for
    a $4 \times 4$ vector size.}
  \label{table3}

  \begin{tabular}{|c|c|} \hline
    Level & Edge Vectors \\ \hline

    & (11)\\
    & (12)\\
    L1 & (13)\\
    & (14)\\
    & (15) \\ \hline

    & (7,0)\\
    & (7,1)\\
    & (7,2)\\
    & (7,6)\\
    L2 & (8,4)\\
    & (8,5)\\
    & (8,6)\\
    & (9,6)\\
    & (9,14)\\
    & (10,1) \\ \hline

    & (4,6,14)\\
    & (5,6,6)\\
    & (6,14,0)\\
    & (6,14,3)\\
    L3 & (6,14,4)\\
    & (6,14,5)\\
    & (7,7,0)\\
    & (7,14,7)\\
    & (9,5,3)\\
    & (9,5,10)\\
    & (9,5,11) \\ \hline

  \end{tabular}

\end{table}

\begin{table}[!t]
  \renewcommand{\arraystretch}{1.3}
  \centering

  \caption{ List of edge vectors for a codebook with b=16 and d=3, for
    a $2 \times 2$ vector size.}
  \label{table4}

  \begin{tabular}{|c|c|} \hline
    Level & Edge Vectors \\ \hline

    & (9)\\
    & (10)\\
    L1 & (11)\\
    & (12)\\
    & (13) \\ \hline

    L2 & (6,0)\\
    & (6,3) \\ \hline

    & (2,2,8)\\
    & (6,5,1)\\
    & (6,5,4)\\
    & (6,5,6)\\
    & (6,5,7)\\
    & (6,5,8)\\
    L3 & (6,5,15)\\
    & (7,0,14)\\
    & (8,0,1)\\
    & (8,15,3)\\
    & (8,15,4)\\
    & (8,15,10) \\ \hline

  \end{tabular}

\end{table}

\begin{table}[!t]
  \renewcommand{\arraystretch}{1.3}
  \centering

  \caption{ List of edge vectors for a codebook with b=16 and d=3, for
    a $6 \times 6$ vector size.}
  \label{table5}

  \begin{tabular}{|c|c|} \hline
    Level & Edge Vectors \\ \hline

    & (6)\\
    & (7)\\
    & (8)\\
    & (9)\\
    L1 & (10)\\
    & (11)\\
    & (12)\\
    & (13)\\
    & (14)\\
    & (15) \\ \hline

    & (2,8)\\
    & (2,13)\\
    & (4,1)\\
    & (4,6)\\
    L2 & (4,7)\\
    & (4,8)\\
    & (4,10)\\
    & (4,11)\\
    & (4,13)\\
    & (4,15) \\ \hline

    & (1,7,0)\\
    & (1,7,1)\\
    & (1,7,2)\\
    L3 & (1,7,3)\\
    & (1,7,4)\\
    & (1,7,6)\\
    & (1,7,9)\\
    & (1,7,12) \\ \hline

  \end{tabular}

\end{table}

\appendix{Another Example Appendix}


\appendixsection{Background}
\label{sec:back}


Some random appended text for this section of the appendix....


\appendixsection{Meat of the Appendix}
\label{sec:meat}

Here we have the data that is so important to be included in this
appendix.
